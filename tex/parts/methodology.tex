\chapter{Methodology}

\section{Introduction}
As stated earlier in the document, Unauthentic academic documents cause a great deal of trouble in the employment industry and a country's economy. The need for a more secure way of dispatching academic results is the major drive to the operation of this project. \\~\\

This chapter contains a description of the techniques, methods and tools used during the research process. Also contained in this chapter, is a description of the implementation, testing and validation of the system. 

\section{Data Collection}
The data collection process in this project was done through interviews with the various stakeholders of the system for example the students and administrative assistants at different colleges of the university. Document reviews were also done in the process of data collection.\\~\\
From the vast number of stakeholders, sampling was done using the systematic sampling technique. This is a type of probability sampling where every element of the sample space has an equal chance of being picked. 

\subsection{Interviews}

This is a one to one discussion between the project team and the expected users of the system. An advantage this method has over many others is that the interviewer gets first-hand information. This method is also qualitative in nature and  helpful in validating the already gathered information. However, there is a possibility of the interviewee giving false information regarding particular aspects based on their emotional state. \\~\\
The interviews were carried through a one on one discussion with the interviewees.\\~\\

\textbf{Reasons for using this method}
\begin{itemize}
\item  Quick feedback from respondents cuts short on the time of requirements collections.
\item There is a possibility of asking questions that are not included in the interview script.
\item This technique allows respondents to describe what is more important to them.
\end{itemize}

\subsection{Observation}
As a source of additional information, we carried out some observations of the current results management system used by the university. This, we did by accessing the student portal(ar.mak.ac.ug)used for results. \\~\\
The major aim for our observations was to get qualitative information about the existing system that handles students’ results.\\~\\

\textbf{Reasons for using this method}
\begin{itemize}
\item Provides access to situations and people where questionnaires and interviews are inappropriate to use.
\item Strong on validity and in-depth understanding of the design problem.
\item Good for explaining meaning and context.
\end{itemize}


\subsection{Document Review}
Through literature review, we collected information from already existing related research. This research was done on similar systems to the one involved in this project that already exist. This information was gathered from online sources like the internet, journals an and other relevant materials on the problem domain from libraries around the university.

\section{Tools}
We used notebooks and pens to note down information obtained during the interviews. I addition, we also used smartphones to record the conversations. tabulated below is a summary of the methods used in the data collection process and tools used to implement them.\\~\\

\begin{table}
\caption{Methods and tools}
\begin{tabular}{|p{4cm}|p{10cm}|}
\hline
\textbf{Method}&\textbf{Tools}\\
\hline
\hline
Interviews & \begin{itemize}
\item Interview guide
\item Pens and notebooks
\item Smartphones
\end{itemize}\\ 
\hline

Observations & \begin{itemize}
\item Personal Computers
\item Smartphones
\item Internet
\end{itemize}\\
\hline
Document Review & \begin{itemize}
\item Books
\item Articles
\end{itemize}\\

\hline

\end{tabular}
\end{table}


\section{Data Analysis}
The collected data was analyzed to be able to attain consistency and reliability for proper modeling and implementation of the system. The data was studied to identify key user and system requirements. These were classified under functional and non-functional requirements.

\section{System Design and Implementation}
In order to achieve efficient project management, we adapted the agile software development methodology where the requirements and solutions evolve together throughout the development process. This approach significantly contributed to the success of the project because we were able to anticipate the need for flexibility in time.

\subsection{System Design}
The system was designed using use case and data flow diagrams. The generated use case diagram visually express how the different users interact with the system and the data flow diagrams showed how data flows through the system.\\~\\
In the development of the use cases, we used the object-oriented analysis development method to design the system. In this phase, we determined the system’s requirements and identified the classes and the relationships among the different classes that use the system. There are three major analysis techniques used together during the object oriented analysis namely;
\begin{itemize}
\item \textbf{bject modeling:} This involves developing the static structure of the software system in terms of objects. 
\item \textbf{Dynamic modeling:} This examines the behavior of the system with respect to time and external changes after the static behavior of the system has been analyzed.
\item \textbf{Functional modeling:}This shows the processes that are performed within an object and how the data changes as it moves between methods.
\end{itemize}

\subsection{System Implementation}
The buiding of the system started after having the design complete. this was done by writing code. The tasks were further divided among the different individuals working on the project.  Each developer followed a predefined set of guidelines for collaboratin during the development process. \\~\\

JavaScript and Solidity which is a language used to build smart contracts on the Ethereum platform, are some of the languages used.
Another important software in this development is ganache. Ganache is a software that allows one to simulate a blockchain locally on a computer. Ganache has a number of features including displaying the accounts on the local blockchain, the transactions made, the blocks in the network, among others.\\~\\
Finally, we use MetaMask to run the application on a browser. MetaMask is a google chrome extension that allows one run a blockchain application on google chrome browser. 

\section{Testing and Validation}
\subsection{System Testing}
This phase involves the assessment of the system to verify if it works properly and also verify if it satisfies the specified requirements. The system has been availed to potential users to test it. This enables the team get feedback from the potential users of the system.
There are two major approaches to testing of the system used ie;
\begin{itemize}
\item \textbf{White box testing:} In this technique, we critically studied the source code to find out which unit or chunk of code is behaving inappropriately. This helped in optimizing the code and removing extra lines of code which bring in hidden defects or adding more lines of code to make the application work even better.

\item \textbf{Black box testing:} in this technique, a tester interacts with the system’s user interface by providing inputs and examining outputs without knowing how and where the inputs are worked upon. 
\end{itemize}

\subsection{Validation}
Systems validation is the process of checking that a software system meets specification and it fulfills its intended purpose. To ensure data quality, errors should be detected during input, prior to processing and storage and this will be achieved through validating input transactions and input data. If the system conforms to the specified user requirements, the first release will finally be deployed. With time, the system will be upgraded with more improvements and innovative features; this is because systems without innovative features lose their usability in the long run.

\section{Conclusion}
To realize all this, we had to put into play our project management skills and this called for the need to track our progress using daily Stand Up meetings. This is recorded and shared on google sheets \cite{art12}.


