\chapter{Methodology}

\section{Introduction}
As stated earlier in the document, Unauthentic academic documents cause a great deal of trouble in the employment industry and a country's economy. The need for a more secure way of dispatching academic results is the major drive to the operation of this project with a focus on Makerere University. \\~\\
This chapter contains a description of the techniques, methods and tools used during the research process. Also contained in this chapter, is a description of the implementation, testing and validation of the system. 

\section{Data Collection}
Data collection for this project was approached using both qualitative and quantitative methods. The quantitative methods helped in evaluating the impact of the current results storage methods on the students, employees and other stakeholders involved. On the qualitative end of the card, we were able to obtain opinions on the present day methods of storing students' results. More importantly, learning stakeholders' opinions on a possibility of storing results in a decentralized manner was a major drive to use qualitative methods.
The data collection process in this project was done through interviews with the various stakeholders of the system for example the students and administrative assistants at different colleges of the university. Document reviews were also done in the process of data collection.\\~\\
From the vast number of stakeholders, sampling was done using the systematic sampling technique. This is a type of probability sampling where every element of the sample space has an equal chance of being picked. 

\subsection{Interviews}
Interviews were carried out with administrative assistants at various colleges of the university. These are the university employees in charge of feeding students results into a system where they're managed. This made them a key source of information. Information sought included;
\begin{itemize}
\item Strengths of the current results storage system.
\item Weaknesses of the current results storage system.
\item Opinion on a possibility of a decentralised storage system for the students records.
\end{itemize}
The conversations held during the interviews were in person in the various offices of the different administrative assistants. The conversations with each respondent took approximately thirty minutes excluding the amount of time spent trying to make appointments.\\~\\
\textbf{Relevance of Interviews}
\begin{itemize}
\item Feedback was quickly obtained from the respondents.
\item There was room to get more information than what was intended in the interview script for example, we learnt of a results management and storage system that was used at the College of Computing and Information Sciences for about four years but was never adopted by the university.
\item Since the conversations happened in person, there was room to read the non verbal cues from the respondents. This creates room to assess whatever the respondent is saying from their body language which would rather not be possible with other methods \cite{art13}.
\end{itemize}

\subsection{System Review}
As students of the university, we had access to the current results storage system with privileges granted to students. We carried out a review of the system accessed through the url ar.mak.ac.ug. This method was a source of qualitative information as it's major purpose was to learn the strengths and weaknesses of the system. some of the attributes looked out for were security, availability among others.\\~\\
\textbf{Relevance of a System Review}
\begin{itemize}
\item We were able to obtain first hand information from the system itself.
\item Access to the source of information (The students' portal) is much easier as it is available for us to access.
\end{itemize}


\subsection{Document Review}
Through literature review, we collected information from already existing related research. This research was done on already existing block chain applications similar to the one this project is focused on for example Blocksign, Siacoin, Storj and Cryptyk. This information was gathered from online sources like the internet, journals and other relevant materials on the problem domain from libraries around the university.


\section{Data Analysis}
At this stage of the project, a vast amount of data had been collected and the team was then tasked to filter out the relevant data. A great deal of the data collected was qualitative. This therefore called for qualitative data analysis techniques. Our findings were examined against a predefined framework based on the objectives of the project. The major method used is the Inferential data analysis method. This approach studies the relationship between various variables for example the relationship between a student and an employer.\\~\\


\section{System Design and Analysis}
In order to achieve efficient project management, we adapted the agile software development methodology where the requirements and solutions evolve together throughout the development process. This approach significantly contributed to the success of the project because we were able to anticipate the need for flexibility in time. 

\subsection{System Design}
The system was designed using use case and class diagrams. The generated use case diagram visually express how the different users like students interact with the storage system. \\~\\
In the system design process, we used the object-oriented analysis development method. The system’s requirements were determined at this phase. The various classes and relationships among them were also identified at this stage. There are three major analysis techniques used together during the object oriented analysis namely;
\begin{itemize}
\item \textbf{Object modeling:} This involves developing the static structure of the software system in terms of objects. Here we identified the various classes like students, administrative assistants among others. Identified at this stage also include attributes of the various classes, associations and operations performed on them.
\item \textbf{Dynamic modeling:} The major activity at this stage involved identifying relationships between the various classes. Some classes extended others that are abstract.  
\item \textbf{Functional modeling:} Here, we identified how the data within an object changes as the processes performed within it are executed. Identified also were the changes made on the data as it moves from one stage to another.
\end{itemize}

\subsection{System Implementation}
The implementation of the results storage system was done using the agile method of development. This involved continuous iteration  of development i.e. different elements of the system like the file storage system were tested repeatedly through out the lifecycle of the project.\\~\\
For the purposes of managing tasks, we used a method known as scrum which is a category under the agile method of development. Activities in this method included setting up sprints. These sprints covered a time period of one week, each with a set of tasks split among team members to be completed during the sprint. Each team member at the end of the day had to give a report on the activities asigned to him. This we did with daily stand up meetings held using google sheets\cite{art14}.\\~\\
For the split tasks, we used github for collaboration among team members. The detailed description of how github was used is further explained in section 6.2.1.

\section{Testing and Validation}
At this stage of the project, we assessed the system to ensure that it can operate in different environments on multiple platforms and also confirm that it satisfies the specified requirements set in the earlier stages of the project. The results storage system has been availed to potential users so the team can get feedback from them. The process here involved deploying a sample transcript on a decentralised network from one node and being able to access it from another node.\\~\\
The testing was done in two major parts namely functional testing and non-functional testing. These are futher explained below;
\subsubsection{Functional Testing}
In this phase of testing, the system is tested against the requirements set at the beginning of the project. There are four methods under functional testing that are explained below in their order of execution.
\begin{enumerate}
\item \textbf{Unit Testing:} Here, we carried out tests on the individual modules of the system which included the web application that handles the results and the IPFS file handler which stores the transcript on a decentralised network. Tested at this stage is also the smart contract that handles the storage of the transcripts' hash on a blockchain.
\item \textbf{Integration Testing:} The different individual modules that were tested in the above stage were tested when integrated together to ensure that once results are verified, the transcript is stored on a decentralised network and its hash value stored on a blockchain network. A hash value is the return value of a hash function after mapping data of an arbitrary size onto data of a fixed size\cite{art15}.
\item \textbf{System Testing:} At this stage, we tested the entire system with all the modules integrated together for bugs. The goal here was to test for user-expected conditions. 
\item \textbf{Acceptance testing:} Being the final stage, the system was availed to potential users to make sure it works as expected. The major drive here was to ensure that all the project goals and requirements had been met.
\end{enumerate}

\subsubsection{Non-Functional Testing}
The major drive for this phase of the testing process was to ensure the non-functional requirements were met. This was achieved using four major methods explained below in their order of operation.
\begin{enumerate}
\item \textbf{Performance Testing:} The system was tested for different behavior under various conditions. An endurance test was carried out to monitor how the storage system would behave under sustained use.
\item \textbf{Security Testing:} A security test was carried out to ensure the data and information stored in this system is safe from unauthorized access.
\item \textbf{Usability Testing:} Here, the system was tested for usability from a users' perspective. Some of the tested aspects include the Graphical User Interface (GUI), the work flow of the system among other things.
\item \textbf{Compatibility Testing:} Finally, the system was tested in different environments. Different browsers like Google chrome and Mozilla Firefox were used to try to access the block chain network.
\end{enumerate}

\section{Conclusion}
To realize all this, we had to put into play our project management skills and this called for the need to track our progress using daily Stand Up meetings. This is recorded and shared on Google sheets. An illustration of this is shown in Appendix A - figure 7.4.


