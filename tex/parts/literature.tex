\chapter{Literature Review}

\section{Background}

Recording and verifying candidates' credentials can be costly and time-consuming for academia and businesses alike. Through this project, we look to turn to blockchain technology for a solution. But first of all, let us start with a review of the current system.\\

\subsection{Review of Existing System}
At Makerere University, there is a system that has been developed for the purpose of storage of students’ results. The table below summarizes a review of the existing system.

\begin{tabular}{|p{3cm}|p{4cm}|p{7cm}|}
\hline
Results System&Strength&Weakness\\
\hline
\hline
Web URL - www.ar.mak.ac.ug& \textbf{Reliable:} students are able to access their marks online and follow up in case of any fault.& \textbf{Limited access:} Access is limited to the university’s intranet. 

\textbf{Security:} Despite deployment on an intranet, the system is not secure. It is still susceptible to external threats from hackers or malicious software including worms, viruses, and malware. It could also be subject to internal threats from the users, since it doesn’t cater for issues like weak passwords and access control \cite{art4}.
\\ 
\hline

\end{tabular}

\subsubsection{A review of the proposed system}

Considering the flaws of the current system, we take a brief look into how blockchain technology could provide a solution to the above-mentioned weaknesses. \\ \\
Blockchain works like a decentralized ledger, storing information on a global network that is publicly available and should be safe from tampering. Henri Pihkala, founder and CEO of Streamr, a block chain-based platform for live data streams, captures the paradox: \textit{"we make a central place which is decentralized."} \\ \\
It is an interesting technological innovation that has clearly caught the public’s imagination. But how does it matter to the educational ecosystem?\\ \\ 
Any operation as large and distributed as education in 2018 will find a use for blockchain technology. Educational institutions like universities are large-scale, multi-institutional operations that run on the whims and fancies of many moving parts – their stakeholders. Running such institutions is not a simple process, but it can be made simpler through the use of blockchain. Described below are not only some of the contributions, but also weaknesses and gaps that are associated with this technology.

\subsubsection{Contributions}

Because decentralized applications run on the block chain, they benefit from all of its properties, which include:-\\
\begin{itemize}
\item \textbf{Immutability} – A third party cannot make any changes to data.
\item \textbf{Corruption \& tamper proof} – Apps are based on a network formed around the principle of consensus, making censorship impossible.
\item \textbf{Secure} – With no central point of failure and secured using cryptography, applications are well protected against hacking attacks and fraudulent activities.
\item \textbf{Zero downtime} – Apps never go down and can never be switched off.

\end{itemize}

For an example of the contribution of this technology, we look at the University of Nicosia in Cyprus, which is using the technology to record students' achievements. According to George Papageorgiou, a digital currency lecturer at the university, the technology is proving popular. He had this to say to CNBC News: \\ 
\textit{"We've only encountered enthusiasm in the practical uses so far and students are glad to be able to verify, with their new knowledge and the blockchain, that their digital certificate is genuine and that it cannot be recreated.
We believe this instills confidence in both students and potential employers that (they) can check on their own, whether a presented certificate is real or not"}.\cite{art5} \\ \\
This is proof that the implementation is already reaping fruits in some institutions around the world.



\subsubsection{Weaknesses and gaps}

However, despite all the possibilities offered by blockchain, there have been various challenges associated with it. \\ \\
We observe challenges in both the perspective of the end-user, and we, the researchers. From the user’s view, according to Donald Clark \cite{art6}, an EdTech entrepreneur and advisor of EdTech companies, some public sector organizations just don’t like the innovation and stick to their institutional silos. This is basically because the technology has not been around for a long time which makes many potential users have doubts about its possibilities. To overcome this, we intend to train the parties in these institutions on how to use this technology and also show them the advantages. \\ \\
From our research perspective, the major challenge is that the subject of study is of a relatively early stage. Blockchain technologies are under active development globally, and there may be recent advances that impact our findings. \\ \\

In conclusion, it is important to note that blockchain is a technology that clearly has applications in the world of learning at individual, institutional and international levels. It is relevant in all sorts of contexts: schools, colleges, universities among others. \\ \\
One thing we know for sure is that students have their eyes open and are looking for alternatives. Perhaps, like Bitcoin, the blockchain revolution will ultimately come from left of field.\\ \\

\newpage

