\chapter{Recommendations, Conclusion and Future Works}

\subsection{Conclusions}
In conclusion, the major tasks set at the beginning of the project have been completed. All the elements of the students results' storage system have been built and put together to achieve the desired goal. Below, we summarize the progress of the project with respect to the specific objectives.\\~\\
The first specific objective is to ensure safety of students' records. The results storage system stores the students transcript on decentralized network and can only be accessed by those who have the hash value for that transcript. Furthermore, the hash value for the transcript is stored on a blockchain network which has proven immutability characteristics.\\~\\
The other objective is to create an open distributed catalog that makes it easy for students to share their transcripts. This system stores the transcript on a distributed network using a protocol known as IPFS. With this, the student can still share his/her transcript without worry of insecurity.\\~\\
The third objective is to provide a medium for verification that can be used by employers in telling true transcripts from counterfeits. With a QR code appearing on the transcript, an employer is able to scan the document and verify its authenticity.\\\\
Considering a few days between the deadline of this report and the one of the presentation, certain aspects will be improved upon. These include increased level of detail and well designed poster for the presentation.

\subsubsection{Limitations}
During the research and lifetime of the project, a number of limitations were encountered. These were not just on the project but a possibility of the success of a decentralized method of results storage. These limitations are further explained below.\\\\
\textbf{Limited Accounts: }During the developments process, we only had 10 blockchain accounts to use while running tests. This greatly limited us from testing the ultimate potential of a distributed network.\\\\
\textbf{Complexity:} Blockchain is a new technology with a lot of new vocabulary. Although it has given room for cryptography to go mainstream, it is no place for beginners.\\\\
Since this results storage system is based on blockchain, a limitation to this is that many students at Makerere University are not well versed with the present day system that stores their results. Presenting blockchain to the university community could pose a greater challenge. Thankfully, there are a number of courses on blockchain out there today. \\\\
\textbf{Transaction Costs: }In the early stages of a decentralized network, transactions are usually faster due to the small number of nodes. As the number of nodes grow, the network grows along. Using the networks becomes more costly in terns of resources.\\\\
With time, a distributed results storage system like this one would grow. More students and employees would join the network over time. However, a major characteristic of blockchain is its scalability. This would be a good remedy for such a challenge.

\subsection{Recommendations}
With the what we have learned during the research process, we believe certain changes would be helpful for the success of such future projects.
\begin{itemize}
\item We recommend the university to create post project mentor-ship program. This would encourage students to focus on their enhancing their skills past their time of study.
\item We recommend the College of Computing and Information Sciences to prepare students from the time they take on the course for their final year project. We believe this would reduce the pressures of doing the project alongside other course units in two semester. 
\end{itemize}

\subsection{Future Works}
Many tests and experiments for this project have been left out due to limited availability of time and resources. Some of these include running tests on a real blockchain network and not just a local environment. \\\\
The scope of this project has been on building a results storage system with focus on one institution (Makerere University). Future works on this project will aim at expanding the storage network to more than one tertiary institution. Furthermore, the idea is to expand storage of students academic records right from the level in which they begin school to their level of completion. i.e. from lower to higher levels of education.