\chapter*{Abstract}
Students' final academic records (transcripts) in various institutions of learning are issued on paper. This creates room for some dishonest people to create counterfeit transcripts showing results they did not attain. Institutions of course have labels like stamps, seals among others to show that these documents are authentic. However, con-men have mastered the art of replicating these labels to look a lot like the original ones. In the end, many incapable people flood the employment industry with counterfeit academic records. Besides the problem of un-authentic results, there is lots of bureaucracy involved in obtaining the original copy of the transcript.\\\\
Our research findings indicate that sharing and storage of academic records on a distributed network and sharing them remotely would close the gap that allows development of counterfeit documents. This gives an alternative to having a hard copy transcript that is attained faster. It also limits the process of having to verify academic records repeatedly. Furthermore, it becomes easier for employers to verify a potential employee's records.\\\\
Storage of Academic Documents using Blockchain arose after the realization of the bureaucracy and loopholes described above. The system contains a web application that manages students' progressive academic records in a centralized manner and when the transcript is verified by the institution's authority, it is stored on a decentralized network using a protocol known as IPFS (Inter Planetary File Systems). On this network, each academic record has a unique identifier known as a hash value. This hash value is then stored on a blockchain network which itself is decentralized.\\\\
Transcripts are very important in the job market in the world today and having many counterfeits flooding the industry can be fatal to the economy. This research highlights the problems involved in the issuance of these transcript with Makerere University as the institution of focus. Further covered are the solutions to the challenges and how they were developed.

